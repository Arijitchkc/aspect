\documentclass[12pt]{article}
\usepackage{hyperref}
\usepackage{alltt}
\usepackage[margin=1in]{geometry}
\begin{document}

\title{Compiling and Running ASPECT on TACC Stampede}
\author{D. Sarah Stamps \textless\href{dstamps@ucla.edu}{dstamps@ucla.edu}\textgreater \\
and Jonathan Perry-Houts \textless\href{jperryh2@uoregon.edu}{jperryh2@uoregon.edu}\textgreater\\
updated from previous document by Eric Heien published on October 9, 2014}

\maketitle



\noindent Setting up ASPECT on TACC Stampede is much easier than on previous systems since
most of the necessary libraries are already correctly installed and configured. Following all
the instructions in this document should allow you to start running ASPECT within 30
minutes. For further information on Stampede you can also refer to the user guide at \\\url{http://www.tacc.utexas.edu/user-services/user-guides/stampede-user-guide} \\

\noindent Also, please note that most parts of this document have changed since the previous
version. Even if you have set up ASPECT before, please follow all sections of this document
to make sure your current setup has all necessary updates.


\section{Account}
You first need to obtain an XSEDE account at \url{http://portal.xsede.org/}. If you need one send an email to: \begin{alltt}\footnotesize cig-help@geodynamics.org\end{alltt}.  

\section{Setup}
\begin{enumerate}
\item To start edit your $\sim$/.bashrc file to load the correct modules. Update the file with the following in the 
section that says PLACE MODULE COMMANDS HERE and ONLY HERE: 

\begin{alltt}
\footnotesize
  module load git 
  module load gcc/4.7.1 
  module load mkl/13.0.2.146 
  module load cmake 
\end{alltt}

To ensure the setup is correct log out, log in again, and run the following:

\begin{alltt}
\footnotesize	login1.stampede$ module list
\end{alltt}
The result should include all the modules above as well as some default modules:

\begin{alltt}
\footnotesize
Currently Loaded Modules:
  1) TACC-paths      4) xalt/0.4.6   7) git/1.8.5.1     10) mkl/13.0.2.146
  2) Linux           5) cluster      8) gcc/4.7.1       11) cmake/3.1.0
  3) cluster-paths   6) TACC         9) mvapich2/1.9a2
\end{alltt}

\item Get trilinos on Stampede for compiling locally. 

\begin{alltt}
\footnotesize  mkdir \$HOME/Downloads/ \end{alltt} 
Download a copy of trilinos from trilinos.org. When creating this document we used 
 trilinos-11.12.1-Source.tar because it's a known working version for the dealii-8.2.1 
 version we install. You can use other trilinos versions, but doing so requires updating 
 the script aspect-setup.sh with the version you choose.\\

 \begin{alltt}\footnotesize
 scp trilinos-11.12.1-Source.tar  login@stampede.tacc.utexas.edu:/work/02668/login/Downloads/ 
 \end{alltt} 
 
 \item Obtain and run currently working script for installing trilinos, p4est, and deal.ii  aspect-setup\_stampede.sh 
 
 
Obtain the script with: 
 \begin{alltt}\footnotesize
wget https://github.com/geodynamics/aspect/blob/master/doc/install/aspect-setup_stampede.sh
 \end{alltt}  
 
 \end{enumerate}

 \section{Install trilinos, p4est, and deal.ii}
 
Since some of these libraries require a long time to compile, it may take a long time to
finish if it is done on a login node (which is shared with dozens or hundreds of other users).
It is recommended you log onto a compute node when compiling. To do so, use the following
command: \\

\begin{alltt}\footnotesize
srun -p development -t 1:00:00 -n 16 --pty /bin/bash -l
\end{alltt}


\noindent You will need to modify the script if you choose to install a different version of trilinos.  See comments within the script for additional information.\\
 
 \begin{alltt}\footnotesize
 aspect-setup.sh trilinos 
 aspect-setup.sh p4est 
 aspect-setup.sh deal.ii    
 \end{alltt}
 
 \noindent You should now have the following directory structure in  \begin{alltt}\footnotesize \$HOME/packages:  \\
  
  build  deal.II  p4est  trilinos
\end{alltt}

 \section{Configure and build ASPECT}
 Here we assume you have a cloned version of ASPECT on github.  
 \begin{alltt}\footnotesize
 cd \$HOME/packages 
 git clone https://github.com/your\_github\_account\_name/aspect.git 
 cd aspect 
 git remote add upstream https://github.com/geodynamics/aspect.git 
 git pull upstream master  
 mkdir build 
 cd build 
 cmake ..  
 \end{alltt}

\noindent You will get a lot of warnings, but as long as you see the following you are ready to compile. \\
-- Generating done \\
-- Build files have been written to: \begin{alltt}\footnotesize /home1/02668/login/packages/aspect/build \end{alltt}

 \begin{alltt}\footnotesize
make -j8 
 \end{alltt}

\section{Running}
When testing ASPECT, it can be useful to run in interactive mode on the cluster. This way you can use multiple processors for fast testing, but still make changes to files and test different options without needing to wait in the batch queue after each change. To start an interactive shell with multiple processors, use: 

 \begin{alltt}\footnotesize
srun -p development -t <maximum time> -n <\#cores> --pty /bin/bash -l 
 \end{alltt}
 
\noindent For example: 
 \begin{alltt}\footnotesize
srun -p development -t 0:30:00 -n 4 --pty /bin/bash -l 
 \end{alltt}
 
\noindent Once logged in you can run ASPECT in parallel using the command:
 
  \begin{alltt}\footnotesize
ibrun ./build/aspect <parameter file> 
  \end{alltt}
 
\noindent The development queue only allows small short runs ($<$2 hours, $<$256 cores). To perform larger runs, you submit a job to the SLURM scheduler. The format of the submission script is detailed in the Stampede user guide (http://www.tacc.utexas.edu/user-services/ user-guides/stampede-user-guide). An example is included below: \\
 
   \begin{alltt}\footnotesize
\#!/bin/bash 
\#SBATCH -J aspect\_run   \# job name 
\#SBATCH -o aspect\_run.o\%j   \# output and error file name (\% expands to jobID) 
\#SBATCH -n 32  \# total number of mpi tasks requested 
\#SBATCH -p normal \# queue (partition) -- normal, development, etc. 
\#SBATCH -t 12:00:00 \# run time (hh:mm:ss) - 12 hours 
ibrun \$WORK/aspect/build/aspect \$WORK/cookbooks/shell\_simple\_3d.prm 
  \end{alltt}

The script is submitted to the run queue using: \\
 
 \begin{alltt}\footnotesize
sbatch <script file> 
  \end{alltt}

\section{File Transfer}
Once a simulation is finished, you may wish to transfer the resulting output files to a local machine for further analysis. There are two common methods of doing so, scp or Globus. scp (secure copy) works best for smaller transfers (1GB or less). The syntax is similar to normal copies: \\

 \begin{alltt}\footnotesize
scp user1@host1:file1 user2@host2:file2 
  \end{alltt}

To recursively copy the contents of a directory, use the -r option. For example, to copy the results of a computation from Stampede to the local directory: 

 \begin{alltt}\footnotesize
scp -r login@stampede.tacc.utexas.edu:/work/01766/login/aspect/output . 
  \end{alltt}

For larger transfers, Globus is recommended since it optimizes transfer rates and tolerates
failures during the transfer. For details about using Globus with Stampede please refer to: 

 \begin{alltt}\footnotesize
https://www.tacc.utexas.edu/user-services/user-guides/stampede-user-guide\#transferring.
  \end{alltt}
 

 



\end{document}
